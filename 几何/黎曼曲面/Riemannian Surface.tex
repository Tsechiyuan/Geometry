\def\allfiles{}
\documentclass{package/fancy-book}
\setlength{\parindent}{2em}
\usepackage[UTF8]{ctex}
%%%%%%%%%% Default Package %%%%%%%%%%%%%
\usepackage{package/color-env}
\usepackage{package/quiver}
\usepackage{background}
\usepackage[object=vectorian]{pgfornament} %% used in title.tex
\usepackage{calligra} %%% (optional) to make the Title text beautiful 
\usepackage{lipsum}  %% for dummy text 
\usepackage[tbtags]{amsmath}
\usepackage{amssymb,amsfonts}  %%% for maths
\usepackage{datetime}

%%%%%% Optional Packages %%%%%%%
\usepackage{lettrine} %% for nice looking 
\usepackage{GoudyIn} %% first Letter of the paragraph
\renewcommand{\LettrineFontHook}{\color{black}\GoudyInfamily{}}
\LettrineTextFont{\itshape}
\setcounter{DefaultLines}{3}%
%%%%%%%%%%%%%%%%%%%%%%%%%%%%%%%%%%%%%
\usepackage{datetime2}
\usepackage{fourier-orns}
\newcommand{\ornamento}{\vspace{2em}\noindent \textcolor{darkgray}{\hrulefill~ \raisebox{-2.5pt}[10pt][10pt]{\leafright \decofourleft 
\decothreeleft  \aldineright \decotwo \floweroneleft \decoone   \floweroneright 
\decotwo \aldineleft\decothreeright \decofourright \leafleft} ~  \hrulefill \\ \vspace{2em}}}



%%%% Bibliography %%%%%%%%%
% Required packages are included in notes class
% Can be tweaked in the notes.cls file itself
\addbibresource{resource/references.bib}
\includeonly{title,1,2,3}
\begin{document}
\include{title.tex}
\backgroundsetup{contents={}} %% to remove background and watermark from other pages
\tableofcontents

\quad
\newpage
这是笔者于2024年本科四年级下学期学习黎曼曲面的学习笔记。同时也是根据授课老师王嘉项老师的课堂板书整理的笔记.

我们假定读者拥有基础的复分析基础和抽象代数知识。微分流形知识不是必要的,但是了解微分流形将会使得理解本课程更容易.如果读者熟练掌握同调代数,这门课程中的上同调内容将会非常容易.
\include{1}
\include{2}
\ifx\allfiles\undefined

	% 如果有这一部分另外的package,在这里加上
	% 没有的话不需要
	
	\begin{document}
\else
\fi
\chapter{Morse不等式的解析证明}
本节我们讨论Witten在上个世纪80年代关于Morse不等式给出的解析证明。%我们讲通过研究流形上的上同调理论以给出Morse不等式,同时讨论这个证明在几何与拓扑两个领域上的重大影响。(论文里面写)
\section{Witten形变与Hodge定理}
设$M$是紧流形.对于$0\leq  i\leq n$的任意整数$i$,令$\beta_i$表示$M$的第$i$个Betti数$\dim (\HdR{1}(M;\R))$。下面的De rham定理说明我们可以用$\beta_i$表示Morse不等式。
\begin{theorem}[De rham定理]
	光滑流形$M$的Derham上同调和$\R$奇异上同调存在自然的同构。
\end{theorem}
该定理的思路是通过微分形式$\omega$在复形$\Delta^k$上的积分,给出$\R$系数奇异上同调类,在说明这是一个同构。具体的证明,参见82.

这个定理表明我们可以解析地研究流形的拓扑。其关键是研究$\Omega^k(M)$在不同复形下的性质。

\subsection{Witten形变}
给定$M$上的Morse函数$f$,对外微分算子$d$做出如下的形变:
\begin{align}
	d_{Tf}=e^{-Tf}d e^{Tf}
\end{align}

考虑$d^2=0$,于是:
\begin{align*}
	(d_{Tf})^2=e^{-Tf}d^2 e^{Tf}=0
\end{align*}

因此一般的de Rham复形可以变形为$(\Omega^*(M),d_{Tf})$.从而我们可以定义起上同调群:
\begin{align*}
	H_{Tf,dR}^*(M,\R)=\frac{\ker(d_{Tf})}{\mathrm{Im}(d_{Tf})}
\end{align*}
其$\Z$分次结构为:
\begin{align*}
	H_{Tf,dR}^*(M,\R)=\bigoplus_{i=0}^n H_{Tf,dR}^i(M,\R)=\bigoplus_{i=0}^n \frac{\ker(d_{Tf}|_{\Omega^i(M)})}{\mathrm{Im}(d_{Tf})|_{\Omega^{i-1}(M)}}
\end{align*}

之所以考虑这样的形变,是因为我们发现形变后得到的上同调群与原来的上同调群维数一致:
\begin{proposition}
	对于$0 \leq i \leq n$的任意整数$i$,有:
	\begin{align*}
		\mathrm{dim}(H^i_{Tf,dR}(M,\R))=\mathrm{dim}(H^i_{dR}(M,\R))
	\end{align*}
\end{proposition}
\begin{proof}
	对于形式$\alpha$,定义映射$i:\alpha \mapsto e^{-Tf}\alpha$.若$\alpha$是闭形式,则$i(\alpha)$满足$d_{Tf}(i(\alpha))=0$.另外,若$\alpha=d \beta$,则$d_{Tf}(e^{-Tf}\beta)=e^{-Tf}\alpha$.从而$d_{Tf}(i(\beta))=i(\alpha)$.

	因此$i$诱导了上同调的映射$i^*:\mathrm{dim}(H^i_{dR}(M,\R))=\mathrm{dim}(H^i_{Tf,dR}(M,\R))$.显然这是一个同构。

\end{proof}
\subsection{Hodge定理}
设$g^{TM}$是$M$的一个黎曼度量。我们给出DeRham复形上的Hodge定理。
\begin{theorem}[Hodge]
	
\end{theorem}
对于任意的$\alpha,\beta \in \Omega^*(M)$,形式的有:
\begin{align*}
	\langle d_{Tf}\alpha,\beta\rangle=\langle e^{-Tf}de^{Tf}\alpha,\beta\rangle=\langle \alpha,e^{Tf}d^*e^{-Tf}\beta
\end{align*}
于是$d_{Tf}$的形式伴随可写为:
\begin{align}
	d_{Tf}^*=e^{Tf}d^*e^{-Tf}
\end{align}
仿照Hodge分解,定义:
\begin{align}
	D_{Tf}&=d_{Tf}+d^*_{Tf}\\
	\square_{Tf}&=D_{Tf}^2=d_{Tf}d^*_{Tf}+d_{Tf}^*d_{Tf}
\end{align}
从而可知$\square_Tf$保持$\Omega^i(M)$的次数。因为这样的操作并没有改变Hodge定理的关键部分,所以我们可以给出形变后的Hodge定理:
\begin{align}
	\mathrm{dim}(\ker(\square_{Tf}|_{\Omega^i(M)}))=\mathrm{dim}(H^i_{Tf,dR}(M,\R))=\mathrm{dim}(H^i_{dR}(M,\R))
\end{align}
上式说明,为了考量$\beta_i$的信息,可以研究$\square_{Tf}$的行为。因为$T$值本身不改变上述结果,从而可以考虑$T$值趋近于无穷的情况。
\section{算子在临界点的分析}
不失一般性,假设$f$在临界点$x \in M$的开邻域$U_x$上有坐标$(y^i)$满足Morse引理,并且:
\begin{align}
	g^{TM}=(dy^1)^2+\dots+(dy^n)^2
\end{align}
\begin{proposition}
   在上述假设下,可以计算$d_{Tf}$和$d_{Tf}^*$.
    \begin{align}
	d_{Tf}=d+Tdf\wedge, \quad d_{Tf}^*=d^*+Ti_{(df)^*}
    \end{align}
其中$(df)^*$是$df$作为微分形式在度量$g^{TM}$的对偶。
\end{proposition}
\begin{proof}
	设$\omega \in \Omega^*(M)$.则:
	\begin{align}
		d_{Tf}(\omega)=e^{-Tf}(e^{Tf}d\omega+Te^{Tf}df\wedge \omega)=d\omega+Tdf\wedge \omega
	\end{align}
	由于$df\wedge$的形式伴随是$i_{(df)^*}$,于是$d_{Tf}^*$的表达式如上。
\end{proof}
于是可以给出:
\begin{align}
	D_{Tf}=D+T\hat{c}(df),\hat{c}(df)=df\wedge+i_{(df)*} \text{是}df\text{的Dirac算子}
\end{align}
根据Morse引理,在每个$U_x$上都有:
\begin{align}
	df(x)=-y^1dy^1-\dots-y^{n_f(x)}dy^{n_f(x)}+y^{n_f(x)+1}dy^{n_f(x)+1}+\dots+y^ndy^n
\end{align}
其中$n_f(x)$是$f$在$p$处的指数。设$e_i=\frac{\partial}{\partial y^i}$是$TU_x$的定向标准正交基础,则在$U_x$上,有:
\begin{align}
	\square_{Tf}&=-\sum_{i=1}^n (\frac{\partial}{\partial y^i})^2-nT+T^2\|y\|^2+T\sum_{i=1}^{n_f(x)}(1-c(e_i)\hat{c}(e_i))+T\sum_{i=n_f(x)+1}^n(1+c(e_i)\hat{c}(e_i))\\&=-\sum_{i=1}^n (\frac{\partial}{\partial y^i})^2-nT+T^2\|y\|^2+2T(\sum_{i=1}^{n_f(x)}i_{e_i}e_i^*\wedge+\sum_{i=1}^{n_f(x)+1}e_i^*\wedge i_{e_i})
\end{align}

不难发现算子:
\begin{align*}
	\sum_{i=1}^{n_f(x)}i_{e_i}e_i^*\wedge+\sum_{i=1}^{n_f(x)+1}e_i^*\wedge i_{e_i}
\end{align*}
非负,且具有由下述形式生成的一维核空间:
\begin{align*}
	dy^1\wedge \dots \wedge dy^{n_f(x)}
\end{align*}

因此总结如下:
\begin{proposition}
	对于任意$T>0$,算子$\square_{Tf}$作用在$\Gamma(\bigwedge^*(E_n^*))$上非负,其由下述形式生成:
	\begin{align}
		\exp(\frac{-T\|y\|^2}{2})dy^1\wedge \dots \wedge dy^{n_f(x)}
	\end{align}
	进而对于某个固定的常数$C>0$,该算子所有的非零特征值大于$CT$.
\end{proposition}
\section{Morse不等式的证明}
\subsection{一个命题}
我们先不加证明的指出下面命题,并借此证明Morse不等式。
\begin{proposition}\label{property3.3}
	对于任意$c>0$,存在$T_0>0$使得当$T\geq T_0$时,$\square_{Tf}|_{\Omega^i(M)}$的落在区间$[0,c]$的特征值的个数(计重数)为$m_i,0 \leq i \leq n$.
\end{proposition}


对于$0 \leq i \leq n$的任意整数$i$,设:
\begin{align*}
	F_{Tf,i}^{[0,c]}\subset \Omega^i(M)
\end{align*}
是$\square_{Tf}|_{\Omega^i(M)}$在$[0,c]$中的特征值对应的特征向量所生成的向量空间。根据上述命题\ref{property3.3},这个向量空间是$m_i$维的。

通过简单的计算,可以注意到下面的事实:
\begin{align*}
	[d_{Tf},\square_{Tf}]=[d^*_{Tf},\square_{Tf}]=0
\end{align*}
所以$d_{Tf}$和$d_{Tf}^*$对于$F_{Tf,i}^{[0,c]}$而言都是封闭的。($d_{Tf}$把$F_{Tf,i}^{[0,c]}$映射到$F_{Tf,i+1}^{[0,c]}$,$d^*_{Tf}$把$F_{Tf,i}^{[0,c]}$映射到$F_{Tf,i-1}^{[0,c]}$)

因此我们构造了$(\Omega^*(M),d_{Tf})$有限维的子复形:
\begin{align}
	(F_{Tf}^{[0,c]},d_{Tf}):0 \longrightarrow F_{Tf,0}^{[0,c]} \longrightarrow F_{Tf,1}^{[0,c]} \longrightarrow \dots F_{Tf,n}^{[0,c]} \longrightarrow 0
\end{align}

将$\Omega^*(M)$的Hodge分解限制在该有限维复形上,可以给出:
\begin{align}
	\beta_{Tf,i}^{[0,c]}:=\dim (\frac{\ker(d_{Tf}|_{F_{Tf,i}^{[0,c]}})}{\mathrm{Im}(d_{Tf}|_{F_{Tf,{i-1}}^{[0,c]}})})
\end{align}
就等于$\mathrm{dim}(\ker(\square_{Tf}|_{\Omega^i(M)}))$,从而等于$\beta_i$.由于$\beta_{Tf,i}^{[0,c]}$总是小于$m_i$的,因此弱Morse不等式证毕。

下面我们说明强Morse不等式。具体写出$F_{Tf,i}^{[0,c]}$的分解:
\begin{align}
	\dim(F_{Tf,n}^{[0,c]})&=\dim(\ker(d_{Tf}|_{F_{Tf,i}^{[0,c]}}))+\dim(\mathrm{Im}(d_{Tf}|_{F_{Tf,i}^{[0,c]}}))\\&=\dim(\frac{\ker(d_{Tf}|_{F_{Tf,i}^{[0,c]}})}{\mathrm{Im}(d_{Tf}|_{{F_{Tf,i-1}}^{[0,c]}})})+\dim(\mathrm{Im}(d_{Tf}|_{F_{Tf,i-1}^{[0,c]}}))+\dim(\mathrm{Im}(d_{Tf}|_{F_{Tf,i}^{[0,c]}}))
\end{align}
结合上述分解和命题\ref{property3.3},可以得到:
\begin{align}
	&\sum_{j=0}^i (-1)^j m_{i-j}\\=&\sum_{j=0}^i(\beta_{i-j}+\dim(\mathrm{Im}(d_{Tf}|_{F_{Tf,i-j-1}^{[0,c]}}))+\dim(\mathrm{Im}(d_{Tf}|_{F_{Tf,i-j}^{[0,c]}})))\\=& \sum_{j=0}^i (-1)^j \beta_{i-j}+\dim(\mathrm{Im}(d_{Tf}|_{F_{Tf,i}^{[0,c]}}))
\end{align}

可见这就是强Morse不等式。若$i=n$,可知上式的最右边一项为$0$,于是给出取等条件。
\subsection{命题\ref{property3.3}的证明:Witten形变算子}
解析证明的灵魂是命题\ref{property3.3}.为了说明这个命题,我们需要用到不少解析的技巧。

首先做两个记号.对于$T>0$和$f$的临界点$x \in M$,定义
\begin{align*}
    \alpha_{x,T}=\int_{U_x}\gamma(|y|^2)\exp(-T|y|^2)dy^1\wedge \dots \wedge dy^n\\
	\rho_{x,T}=\frac{\gamma(|y|)}{\sqrt{\alpha_{x,T}}}\exp(\frac{-T|y|^2}{2})dy^1\wedge \dots \wedge dy^{n_f(x)}
\end{align*}

其中$\gamma$是选定的一个光滑函数$\gamma:\R \to \R$。满足$\gamma(x)=1,|x|\leq a,\gamma(x)=0,|x|\geq 2a$.这样的光滑函数的存在性是基础的。

于是在上述记号下,$\rho_{x,T}$是在Hodge范数下的单位向量,且拥有在$U_x$里面的紧支撑集。

固定$T$,从而每个临界点$x$处都存在一个$\rho_{x,T}$.用$E_T$表示这些$\rho_{x,T}$生成的有限维向量空间。(因为$x$的个数是有限的)这个向量空间是Sobolev空间$\mathbb{H}^0(M)$的子空间。其中,$\mathbb{H}^0$是$\Omega^*(M)$结合Hodge内积给出的Hilbert空间.

因此可以考虑正交补$E_T^{\perp}$.
\begin{align*}
	\mathbb{H}^0(M)=E_T \oplus E_T^{\perp}
\end{align*}

再设$p_T$和$p_T^{\perp}$是对应的正交投影。

\begin{definition}[Witten形变算子]
	定义如下四个算子。
	\begin{align}
		D_{T,1}:=p_T D_{Tf}p_T,D_{T,2}:=p_T D_{Tf} p_T^{\perp},D_{T,3}:=p_T^{\perp} D_{Tf}p_T,D_{T,4}:=p_T^{\perp} D_{Tf}p_T^{\perp}
	\end{align}
	
\end{definition}

这四个算子有非常好的估计式:
\begin{proposition}
	对于任意$T>0$,$D_{T,1}=0$.
\end{proposition}
\begin{proof}
	设$f$的临界点集合为$\mathrm{zero}(df)$.则对于任何$s \in \mathbb{H}^0(M)$:
	\begin{align*}
		p_T s=\sum_{x \in \mathrm{zero}(df)}\langle \rho_{x,T},s\rangle\rho_{x,T}
	\end{align*}
	我们代入$D_{Tf}$计算:
	\begin{align}
		D_{Tf}(\langle \rho_{x,T},s\rangle\rho_{x,T}) \in \Omega^{n_f(x)-1}(M)\oplus \Omega^{n_f(x)+1}(M)
	\end{align}

	因为上述结果在$U_x$外全为$0$,因此$D_{Tf}(\langle \rho_{x,T},s\rangle\rho_{x,T})$与$\rho_{y,T}$在$x\neq y$的时候做内积显然为$0$.其次,若$x \neq y$,则根据形式的阶数不相同,也得到内积为$0$.综上:
	\begin{align}
		p_T D_{Tf}(\langle \rho_{x,T},s\rangle\rho_{x,T})=0
	\end{align}
\end{proof}
\begin{proposition}
	存在常数$T_1>0$,使得对于任意的$s \in (E_T)^{\perp} \cap \mathbb{H}^1(M)$,$s' \in E_T$和$T \geq T_1$,有:
	\begin{align}
		\|D_{T,2}s\|_0\leq 
	\frac{\|s\|_0}{T},\|D_{T,3}s'\|_0\leq \frac{\|s'\|_0}{T}
	\end{align}
\end{proposition}
\begin{proof}
	容易验证$D_3$是$D_2$的形式伴随。
	\begin{align}
		\langle D_2s,s'\rangle &=\langle p_TD_{Tf}p_T^{\perp}s,s'\rangle\\&=\langle p_T^{\perp}s,D_{Tf}p_Ts'\rangle\\&=\langle s,p_T^{\perp}D_{Tf}p_Ts'\rangle=\langle s,D_{T,3}s'\rangle
	\end{align}
	因此只需要验证第一个不等式。直接计算有
	\begin{align}
		D_{T,2}s=\sum_{x \in \mathrm{zero}(df)}\langle \rho_{x,T},D_{Tf}s\rangle\rho_{x,T}=\sum_{x \in \mathrm{zero}(df)}\langle D_{Tf}\rho_{x,T},s\rangle\rho_{x,T}=\sum_{x\in \mathrm{zero}(df)}\int_{U_x}D_{Tf}\rho_{x,t} \wedge (*s) dv_p \rho_{x,T}
	\end{align}
	根据$\rho_{x,T}$的构造,我们不难发现上述积分只在$|y|\in [a,2a]$的区间上有意义。($\ker D_{Tf}=\ker D_{Tf}^2$.)而此时上述积分的值就取决于$\|s\|_0$和$T$.具体可以描述为:

	存在$T_0>0$,$C_1>0$,$C_2>0$,使得:
	\begin{align}
		\|D_{T,2}s\|_0 \leq C_1T^{n/2}\exp(-C_2T)\|s\|_0
	\end{align}
	于是命题得证。
	\end{proof}
\begin{proposition}
	存在常数$T_2>0$,$C>0$使得对于任何的$s \in E_T^\perp \cap \mathbb{H}^1(M)$和$T\geq T_2$,有:
	\begin{align}
		\|D_{Tf}s\|_0\geq C_2\sqrt{T}\|s\|_0
	\end{align}
\end{proposition}
\begin{proof}
	参照参考文献84页。
\end{proof}

\ifx\allfiles\undefined
	
	% 如果有这一部分的参考文献的话,在这里加上
	% 没有的话不需要
	% 因此各个部分的参考文献可以分开放置
	% 也可以统一放在主文件末尾。
	
	%  bibfile.bib是放置参考文献的文件,可以用zotero导出。
	% \bibliography{bibfile}
	
	\end{document}
	\else
	\fi
\end{document}